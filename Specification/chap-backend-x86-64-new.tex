\chapter{x86-64 (new version)}
\label{chapter-backend-x86-64-new}

\section{Register usage}
\label{sec-backend-x86-64-new-register-use}

Most conventions for register usage are based on the idea that many
registers need to permanently assigned to things like arguments and
return values so that function calls between separately compiled
functions can be as efficient as possible.  But one main feature of
\sysname{} is the call-site manager which uses information about both
the caller and the callee to transmit arguments and return values.  As
a result, the default calling conventions are used only in some
cases.  Therefore, it is more important to make the default calling
conventions as simple as possible.

Furthermore, we do not use any callee-saves registers.  The call-site
manager is able to optimize register use for leaf routines, by
essentially treating every register as a callee-saves register, except
that this optimization is more flexible, since, again, the call-site
manager has knowledge about both the caller and the callee.  By not
using callee-saves registers, we are also able to simplify both the
register allocator, the compiler, and the root-finding phase of the
garbage collector.

As a result of these considerations, most registers can be used as
general-purpose registers for holding lexical variables.  The
following table shows the resulting register usage:

\begin{tabular}{|l|l|l|}
\hline
Name & Used for\\
\hline
\hline
RAX & First return value\\
RBX & Dynamic environment\\
RCX & Register variable\\
RDX & Register variable\\
RSP & Stack pointer \\
RBP & Frame pointer \\
RSI & Register variable\\
RDI & Register variable\\
R8  & Register variable\\
R9  & Register variable\\
R10 & Static env. argument\\
R11 & Register variable\\
R12 & Register variable\\
R13 & Register variable\\
R14 & Register variable\\
R15 & Register variable\\
\hline
\end{tabular}

We use the \texttt{FS} segment register to hold an instance of the
\texttt{thread} class.  Such an instance will be used in a variety of
situations including debugging support, etc.

For information about the call-site descriptor, see
\refSec{sec-call-site-descriptor}.

% LocalWords:  callee
