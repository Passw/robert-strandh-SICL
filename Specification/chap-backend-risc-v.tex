\chapter{\riscv{}}
\label{chapter-backend-risc-v}

\section{Register usage}
\label{sec-backend-risc-v-register-usage}

The \riscv{} manual contains the following table concerning the register
use in the C ABI:

{\small
\begin{tabular}{|l|l|l|l|}
  \hline
  Register & ABI Name & Description & Saver\\
  \hline
  \hline
  x0 & zero & Hard-wired zero & -\\
  x1 & ra & Return address & Caller\\
  x2 & sp & Stack pointer & Callee\\
  x3 & gp & Global pointer & -\\
  x4 & tp & Thread pointer & -\\
  x5-x7 & t0-t2 & Temporaries & Caller\\
  x8 & s0/fp & Saved register/frame pointer & Callee\\
  x9 & s1 & Saved register & Callee\\
  x10-x11 & a0-a1 & Function arguments/return values & Caller\\
  x12-x17 & a2-a7 & Function arguments & Caller\\
  x18-x27 & s2-s11 & Saved registers & Callee\\
  x28-x31 & t3-t6 & Temporaries & Caller\\
  \hline
  f0-f7 & ft0-ft7 & FP temporaries & Caller\\
  f8-f9 & fs0-fs1 & FP saved registers & Callee\\
  f10-f11 & fa0-fa1 & FP arguments/return values & Caller\\
  f12-f17 & fa2-fa7 & FP arguments & Caller\\
  f18-f27 & fs2-fs11 & FP saved registers & Callee\\
  f28-f31 & ft8-ft11 & FP temporaries & Caller\\
  \hline
\end{tabular}}

In \sysname{} we do not use callee-saves registers and we do not use
registers to pass arguments in the default calling conventions, as
explained in \refsec{sec-backend-risc-v-calling-conventions}.  The
corresponding table for \sysname{} becomes:

{\small
\begin{tabular}{|l|l|l|}
  \hline
  Register & Description & Saver\\
  \hline
  \hline
  x0 & Hard-wired zero & -\\
  x1 & Return address & Caller\\
  x2 & Stack pointer & Caller\\
  x3 & Global pointer & -\\
  x4 & Thread pointer & -\\
  x5-x7 & Temporaries & Caller\\
  x8 & Frame pointer & Caller\\
  x9 & Temporary & Caller\\
  x10 & Temporary/ first return value & Caller\\
  x11 & Argument count, value count & Caller\\
  x12-x31 & Temporaries & Caller\\
  \hline
  f0-f31 & FP temporaries & Caller\\
  \hline
\end{tabular}}

\section{Calling conventions}
\label{sec-backend-risc-v-calling-conventions}

These calling conventions are based on the following ideas:

\begin{enumerate}
\item The calling conventions are used only when a call to an unknown
  function is made.  When the call is to a known, globally defined,
  function, the call-site manager bypasses the argument-parsing code
  and stores the arguments directly where the callee expects them.
  Therefore, the performance of the calling conventions is not
  extremely important.
\item The result of evaluating a form for which all values are needed,
  is that the values are pushed on top of the stack.
\end{enumerate}

\subsection{Normal call}

Note that the \riscv{} architecture requires the stack pointer to be
aligned to $16$ bytes.

\begin{enumerate}
\item Compute the callee function and the arguments into a temporary
  locations.
\item Load the rack of the callee into a temporary register.
\item Load the static environment from the rack of the callee into the
  register dedicated for the static environment.
\item Load the initial frame size for the callee from the rack of the
  callee, and put the result in some register, say $rf$.  This number
  is a multiple of $16$.
\item Load the entry point for the callee from the rack of the callee,
  and put the result in some register, say $re$.
\item The number of arguments is known statically, say $A$.  Let $N$
  be $A$ plus $2$ ($1$ for the saved caller frame pointer and $1$ for
  the call-site descriptor), and rounded up so that it is even.
\item Save the stack pointer in some available register, say $rs$.
\item Subtract the contents of $rf + N \cdot 8$ from the stack
  pointer.
\item Store the caller frame pointer on the stack at location $rs - 8$.
\item Store the call-site descriptor on the stack at location $rs -
  16$.
\item Store arguments into locations $sp +  0$, $sp +  1$, etc.
\item Load the dynamic environment into its dedicated register.
\item Copy the $rs$ register into the frame pointer, thereby
  establishing the initial frame for the callee.
\item Call the entry point in $re$.
\end{enumerate}

% LocalWords:  callee
