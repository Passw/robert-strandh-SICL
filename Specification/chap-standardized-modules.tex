\chapter{Modules implementing standardized functionality}

\section{Reader}
\label{sec-reader}

The \commonlisp{} reader is a prime candidate for an
implementation-independent module.  We extracted the \sysname{} reader
to an independent repository under the name \eclector{}%
\footnote{https://github.com/s-expressionists/Eclector}, which was
subsequently greatly improved by Jan Moringen who is now also the
maintainer.

Aside from correctness, one of the main objectives of \eclector{} is
the ability for client code to customize its behavior.  This
customization is accomplished by the following technique:

\begin{itemize}
\item Standard entry points such as \texttt{read} are implemented as a
  simple \texttt{trampoline} to a generic function that takes an
  additional \emph{client} parameter.  The object passed to the
  generic function for this argument is the value of the special
  variable \texttt{*client*}.
\item The main generic function calls other generic functions for
  every aspect of the reader algorithm, each of which also has a
  \emph{client} parameter, so that each of these functions will
  receive the value of the \texttt{*client*} special variable as it
  was when the main entry point was invoked.
\item The default value of the \texttt{*client*} special variable is
  \texttt{nil}, and default methods that implement the reader
  algorithm do not specialize to the \texttt{client} parameter.
\item Client code will bind \texttt{*client*} to some standard object
  of its choice.
\item Clients that need to alter the default behavior of \eclector{}
  can write primary or auxiliary methods on the relevant generic
  functions for the aspect that needs to be customized.  These methods
  will specialize the \emph{client} parameter to the class of their
  chosen standard object.
\end{itemize}

Some examples of the aspects that can be customized are:

\begin{itemize}
\item The interpretation of tokens.  By default, tokens are
  interpreted according to section 2.3 of the \commonlisp{} standard,
  but this behavior is not always wanted by clients.
\item The creation of symbols.  By default, symbols are created
  according to section 2.3.4 of the \commonlisp{} standard, but there
  are many situations where this behavior is not appropriate.  As an
  example, client code may want to avoid that symbols are
  \emph{interned} in its packages.
\end{itemize}

Another main objective of \eclector{} is excellent condition
handling.  A \commonlisp{} reader is an excellent example of one of
the design goals of the \commonlisp{} condition system, i.e., where
low-level code detects a situation that it does not know how to
handle, because there are several possible choices.  The high-level
code that invoked the low-level code must make that decision.  Thus,
\eclector{} signals a condition and proposes one or more
\emph{restarts} that high-level code can invoke.  To make this
mechanism as useful as possible, \eclector{} signals specific
conditions for each such situations, thereby allowing client code to
take the appropriate action.

While an implementation-independent version of the \commonlisp{}
reader was the initial goal of the \sysname{} reader and then of
\eclector{}, we quickly saw other uses for this module.  As it turns
out, many situations require an that behaves in a way similar to the
\commonlisp{} \texttt{read} function, but with some additional twist
to it.  For that reason, \eclector{} contains two additional systems
that take advantage of the design of of this module:

\begin{itemize}
\item A reader producing \emph{concrete syntax trees}.
\item A reader producing \emph{parse results}.
\end{itemize}

As explained in \refSec{sec-concrete-syntax-tree}, a \emph{concrete
  syntax tree} is a data structure that wraps an ordinary S-expression
in a standard object so that additional information can be provided
\emph{about} the S-expression.  In particular, the wrapper object can
contain information about \emph{source location} of the S-expression.

\eclector{} has a system that reads input and returns it as a concrete
syntax tree.  This features contains additional generic functions that
can be customized by client code:

\begin{itemize}
\item Client code can define the representation of a \emph{source
  position}.  By default, \eclector{} calls the function
  \texttt{file-position} and uses the value as a source position.
  Client code can customize this feature in many ways, including the
  use of a Gray stream with its own representation of position.
\item Client code can also define the representation of an
  \emph{interval}, i.e., two source positions, one for the start of an
  expression and another for the end of that expression.  By default,
  \eclector{} creates a \texttt{cons} cell containing the two
  individual positions.
\end{itemize}

The entry point for the standard reader and the entry point for
creating a concrete syntax tree both ignore syntactic features that
are not returned by a conforming \texttt{read} function.  Such
features include \emph{comments} (both line comments and block
comments) and expressions that were excluded based on \emph{reader
  conditionals}.  Sometimes, however, such syntactic features are
important, for example in a text editor that uses the reader to parse
the contents of a \commonlisp{} code buffer.

\section{Printer}
\label{sec-printer}

The printer module, named \incless{} has been extracted to a separate
repository%
\footnote{https://github.com/s-expressionists/Incless}.
It currently provides definitions of standard functions such as
\texttt{print}, \texttt{princ}, \texttt{print-object}, etc.  It does
not directly provide methods on \texttt{print-object} for different
classes of objects, because we want it to be possible for client code
to customize this behavior.  Instead, the default method on
\texttt{print-object} trampolines to a generic function named
\texttt{print-object-using-client} which has an additional
\emph{client} parameter.

The \texttt{format} function is implemented in a different library as
described in \refSec{sec-format}.

\section{Pretty printer}
\label{sec-pretty-printer}

Richard C Waters designed the pretty printer for
\commonlisp{} (\cite{Waters89xp:a}, \cite{Waters:1992:UNC:1039991.1039996}).
It might seem a natural choice to use it for \sysname{}, but we found
that the quality of the code was unacceptably low for the standards of
\sysname{}.  As a result Tarn W. Burton created a completely new
library for pretty printing, called \inravina{}%
\footnote{https://github.com/s-expressionists/Inravina}.  \inravina{}
is perfectly integrated with \incless{}, described in
\refSec{sec-printer}.

\section{Format}
\label{sec-format}

\footnote{https://github.com/s-expressionists/Invistra}

\section{Streams}
\label{sec-streams}

We are using the Cyclosis library%
\footnote{https://github.com/s-expressionists/Cyclosis} for the
\commonlisp{} functions and classes related to streams.  This code was
extracted (with permission) from the Mezzano%
\footnote{https://github.com/froggey/Mezzano}
operating system.

Cyclosis also provides an implementation of Gray streams.

\section{The \texttt{loop} macro} 
\label{sec-loop}

The \texttt{loop} macro is implemented by the extracted system
\khazern{} located in
\texttt{https://github.com/s-expressionists/Khazern}.

\section{High-level functions on lists}
\label{sec-constrictor}

This module is meant to be a complete implementation of portable
functions and macros in the Conses dictionary (chapter 14 in the
\hs{}), except for the low-level functions such as \texttt{cons},
\texttt{car}, \texttt{cdr}, \texttt{rplaca}, and \texttt{rplacd} which
can not be implemented portably.  For its implementation, it uses the
\texttt{loop} macro.  If any other functionality is required, it will
supply special implementations of such functionality, so as to avoid
dependencies on other modules.

We obtain high performance by identifying important special
cases such as the use of \texttt{:test} function \texttt{eq}, or
\texttt{equal}, or the use of a \texttt{:key} of \texttt{identity}.

We supply compiler macros so as to avoid runtime dispatch whenever a
special-case function can be determined by only looking at the call
site.  This ensures high performance for short lists, where argument
parsing would otherwise represent a significant fraction of the cost
of the call.

We are considering whether it might be worthwhile to supply a
macroexpanded version of this module so that no existing
implementation of the \texttt{loop} macro would be required. 

This module is fairly complete, and it includes macros such as
\texttt{push} and \texttt{pop} as well as compiler macros for
functions that take keyword arguments such as the mapping functions. 

The module also includes definitions of specific conditions that are
used by this module, together with condition reporters in English for
those conditions.  It also includes English-language documentation
strings for some of the functions. 

This module is now in a separate repository called \constrictor{}.%
\footnote{https://github.com/s-expressionists/Constrictor}

\section{Type system}
\label{sec-type-system}

Alex Wood implemented \texttt{ctype} which is a complete
implementation of the \commonlisp{} type system, and we use it in
\sysname{}.  In particular, it provides implementations of the
standard functions \texttt{typep} and \texttt{subtypep}.

\section{Sequence functions}
\label{sec-sequence-functions}

This module was written entirely by Marco Heisig.  It provides
high-performance implementations of the functions in the ``sequences''
chapter of the \commonlisp{} standard.  High performance is obtained
by identifying important special cases such as the use of
\texttt{:test} function \texttt{eq}, or \texttt{equal}, or the use of
a \texttt{:key} of \texttt{identity}.  These special cases are handled
by macros according to the technique described by our 2017 ELS paper
\cite{Durand:2017:ELS:Sequence}.

In addition to the technique described in that paper, Marco Heisig
decided to write the sequence functions as generic functions,
specialized to the type of the sequence argument.  Many
implementations have specialized versions of vectors, based on element
type, and a method specialized this way can often be significantly
faster than code that uses a generic \texttt{vector} type.  In order
to account for the different set of vector subclasses available in
different \commonlisp{} implementations, a macro
\texttt{replicate-for-each-vector-class} is used to generate a method
for each such subclass.  Client code can customize this module by
defining this macro according to its set of vector subclasses.

This module can be used as an ``extrinsic'' module, i.e., it can be
loaded into an existing \commonlisp{} implementation without
clobbering the native sequence functions of that implementation.  This
feature has been used to compare the performance of the functions in
this module to that of the native sequence functions of \sbcl{}, and
the result is very encouraging, in that many functions in this module
are as fast, or faster, than the native \sbcl{} equivalents.

The \texttt{sort} functions in this module use the technique described
in a paper by Kim and Kutzner \cite{10.1007/978-3-540-30140-0_63}.
This technique is based on merging.

\subsection{Future work}
\label{sec-sequence-functions-future-work}

Concerning the \emph{sorting functions} (i.e., \texttt{sort} and
\texttt{stable-sort}), there is a challenge.  The current
implementation uses a merging technique where no additional space is
required.  However, the current implementation is not as fast as
traditional merging with O(n) extra space.  So the question is whether
there is an intermediate solution where a small amount of additional
space is used whenever there is such space available, for example on
the stack.

This module has been extracted to a separate repository under the name
\consecution{}.%
\footnote{https://github.com/s-expressionists/Consecution}

\section{Hash tables}
\label{sec-hash-tables}

This module was written by Hayley Patton.  It has been extracted to a
separate repositor under the name \salmagundi{}%
\footnote{https://github.com/robert-strandh/Salmagundi}

\subsection{Package}

The package for all symbols in this chapter is \texttt{sicl-hash-table}.

\subsection{Protocol}

Most of the standard functions on hash tables are implemented as
generic functions:

{\small\Defgeneric {hash-table-p} {hash-table}
}

{\small\Defgeneric {hash-table-count} {hash-table}
}

{\small\Defgeneric {hash-table-size} {hash-table}
}

{\small\Defgeneric {hash-table-rehash-size} {hash-table}
}

{\small\Defgeneric {hash-table-rehash-threshold} {hash-table}
}

{\small\Defgeneric {gethash} {key hash-table \optional default}
}

{\small\Defgeneric {(setf gethash)} {new-value key hash-table \optional default}
}

{\small\Defgeneric {hash-table-test} {hash-table}
}

{\small\Defgeneric {remhash} {key hash-table}
}

{\small\Defgeneric {clrhash} {hash-table}
}

{\small\Defgeneric {maphash} {hash-table}
}

Some additional generic functions are provided, which should be implemented
by a hash table implementation:

\Defgeneric {make-hash-table-iterator} {hash-table}

Return a function which implements the iterator of
\cl{with-hash-table-iterator}.

Furthermore, some functions will be useful for implementing a hash
table:

\Defgeneric {\%hash-table-test} {hash-table}

Return the test function used for comparing keys. This function is necessary
because \cl{hash-table-test} must return a symbol which designates a
standardized test function, and not the function itself; however, an
implementor of a hash table is likely to want to avoid accessing the global
environment when probing keys.

\Defun {find-hash-function} {name}

Return a hash function for the standardized hash table test function
designated by the symbol \cl{name}, or signal an error.

\subsection{Base classes}

\Defclass {hash-table}

This class is the base class of all hash tables.  It is a subclass of
the class \texttt{standard-object}.

\Defclass {hashing-hash-table}

This class provides accessors common for all hash tables which hash
the keys used.  It is a subclass of the class \cl{hash-table}.

\subsection{Hashing hash table protocol}

\Defgeneric {hash-table-hash-function} {hash-table}

\Definitarg {:hash-function}

Return a function which accepts a non-negative fixnum \term{offset value}
and a key object to hash, returning a non-negative fixnum hash.

The hash function defaults to
\cl{(find-hash-function (hash-table-test hash-table))}.

\Defgeneric {hash-table-offset} {hash-table}

\Definitarg {:hash-offset}

The random offset used for hashing with this hash table.  A random
offset is used to avoid an \term{algorithmic complexity attack}, where
an adversary could (indirectly) insert keys that they know will all
collide in the hash table, greatly slowing down an application.  It is
expected that this offset will be used to perturb the hashes
generated, perhaps by being used as the initial state of a hashing
algorithm.

As such, this offset must not be exposed to an untrusted user; but the
offset can be fixed and read for debugging purposes.

\Defgeneric {hash} {hash-table key}

Return the hash of the provided key, specific to the provided hash table.

\subsection{Implementation}

\subsubsection{Hash table implemented as a list of entries}

\Defclass {list-hash-table}

This class is a subclass of the class \texttt{hash-table}.
It provides and implementation of the protocol where the entries are
stored as an association list where the key is the \texttt{car} of the
element in the list and the value is the \texttt{cdr} of the element
in the list.

{\small\Defmethod {gethash} {key (hash-table {\tt list-hash-table})
    \optional default}
}

This method calls the generic function \texttt{contents} with
\textit{hash-table} as an argument to obtain a list of entries of
\texttt{hash-table}.  It also calls the generic function
\texttt{hash-table-test} with \textit{hash-table} as an argument to
obtain a function to be used to compare the keys of the entries to
\texttt{key}.  It then calls the standard \commonlisp{} function
\texttt{assoc}, passing it \textit{key}, the list of entries, and the
test function as the value of the keyword argument \texttt{:test}.  If
the call returns a non-\texttt{nil} value (i.e. a valid entry), then
the method returns two values, the \texttt{cdr} of that entry and
\texttt{t}.  Otherwise, the method return \texttt{nil} and
\texttt{nil}.

\subsubsection{Hash table implemented as a vector of buckets}

\Defclass {bucket-hash-table}

This class is a subclass of \texttt{hash-table}.  The implementation
uses a vector of buckets.

\subsubsection{Hash table implemented using linear probing}

\Defclass {linear-probing-hash-table}

This class is a subclass of \texttt{hash-table}.  The implementation
uses linear probing, based on the cache-aware hash table designed by
Matt Kulukundis and described in a CppCon talk entitled
``Designing a Fast, Efficient, Cache-friendly Hash Table, Step by Step''
\url{https://www.youtube.com/watch?v=ncHmEUmJZf4}.


%%  LocalWords:  subclasses


% LocalWords:  Cyclosis Mezzano
