\chapter{Modules implementing standardized functionality}

\section{Reader}
\label{sec-reader}

The \commonlisp{} reader is a prime candidate for an
implementation-independent module.  We extracted the \sysname{} reader
to an independent repository under the name \eclector{}%
\footnote{https://github.com/s-expressionists/Eclector}, which was
subsequently greatly improved by Jan Moringen who is now also the
maintainer.

Aside from correctness, one of the main objectives of \eclector{} is
the ability for client code to customize its behavior.  This
customization is accomplished by the following technique:

\begin{itemize}
\item Standard entry points such as \texttt{read} are implemented as a
  simple \texttt{trampoline} to a generic function that takes an
  additional \emph{client} parameter.  The object passed to the
  generic function for this argument is the value of the special
  variable \texttt{*client*}.
\item The main generic function calls other generic functions for
  every aspect of the reader algorithm, each of which also has a
  \emph{client} parameter, so that each of these functions will
  receive the value of the \texttt{*client*} special variable as it
  was when the main entry point was invoked.
\item The default value of the \texttt{*client*} special variable is
  \texttt{nil}, and default methods that implement the reader
  algorithm do not specialize to the \texttt{client} parameter.
\item Client code will bind \texttt{*client*} to some standard object
  of its choice.
\item Clients that need to alter the default behavior of \eclector{}
  can write primary or auxiliary methods on the relevant generic
  functions for the aspect that needs to be customized.  These methods
  will specialize the \emph{client} parameter to the class of their
  chosen standard object.
\end{itemize}

Some examples of the aspects that can be customized are:

\begin{itemize}
\item The interpretation of tokens.  By default, tokens are
  interpreted according to section 2.3 of the \commonlisp{} standard,
  but this behavior is not always wanted by clients.
\item The creation of symbols.  By default, symbols are created
  according to section 2.3.4 of the \commonlisp{} standard, but there
  are many situations where this behavior is not appropriate.  As an
  example, client code may want to avoid that symbols are
  \emph{interned} in its packages.
\end{itemize}

Another main objective of \eclector{} is excellent condition
handling.  A \commonlisp{} reader is an excellent example of one of
the design goals of the \commonlisp{} condition system, i.e., where
low-level code detects a situation that it does not know how to
handle, because there are several possible choices.  The high-level
code that invoked the low-level code must make that decision.  Thus,
\eclector{} signals a condition and proposes one or more
\emph{restarts} that high-level code can invoke.  To make this
mechanism as useful as possible, \eclector{} signals specific
conditions for each such situations, thereby allowing client code to
take the appropriate action.

While an implementation-independent version of the \commonlisp{}
reader was the initial goal of the \sysname{} reader and then of
\eclector{}, we quickly saw other uses for this module.  As it turns
out, many situations require an that behaves in a way similar to the
\commonlisp{} \texttt{read} function, but with some additional twist
to it.  For that reason, \eclector{} contains two additional systems
that take advantage of the design of of this module:

\begin{itemize}
\item A reader producing \emph{concrete syntax trees}.
\item A reader producing \emph{parse results}.
\end{itemize}

As explained in \refSec{sec-concrete-syntax-tree}, a \emph{concrete
  syntax tree} is a data structure that wraps an ordinary S-expression
in a standard object so that additional information can be provided
\emph{about} the S-expression.  In particular, the wrapper object can
contain information about \emph{source location} of the S-expression.

\eclector{} has a system that reads input and returns it as a concrete
syntax tree.  This features contains additional generic functions that
can be customized by client code:

\begin{itemize}
\item Client code can define the representation of a \emph{source
  position}.  By default, \eclector{} calls the function
  \texttt{file-position} and uses the value as a source position.
  Client code can customize this feature in many ways, including the
  use of a Gray stream with its own representation of position.
\item Client code can also define the representation of an
  \emph{interval}, i.e., two source positions, one for the start of an
  expression and another for the end of that expression.  By default,
  \eclector{} creates a \texttt{cons} cell containing the two
  individual positions.
\end{itemize}

The entry point for the standard reader and the entry point for
creating a concrete syntax tree both ignore syntactic features that
are not returned by a conforming \texttt{read} function.  Such
features include \emph{comments} (both line comments and block
comments) and expressions that were excluded based on \emph{reader
  conditionals}.  Sometimes, however, such syntactic features are
important, for example in a text editor that uses the reader to parse
the contents of a \commonlisp{} code buffer.

\section{Printer}
\label{sec-printer}

The printer module, named \incless{} has been extracted to a separate
repository%
\footnote{https://github.com/s-expressionists/Incless}.
It currently provides definitions of standard functions such as
\texttt{print}, \texttt{princ}, \texttt{print-object}, etc.  It does
not directly provide methods on \texttt{print-object} for different
classes of objects, because we want it to be possible for client code
to customize this behavior.  Instead, the default method on
\texttt{print-object} trampolines to a generic function named
\texttt{print-object-using-client} which has an additional
\emph{client} parameter.

The \texttt{format} function is implemented in a different library as
described in \refSec{sec-format}.

\section{Pretty printer}
\label{sec-pretty-printer}

Richard C Waters designed the pretty printer for
\commonlisp{} (\cite{Waters89xp:a}, \cite{Waters:1992:UNC:1039991.1039996}).
It might seem a natural choice to use it for \sysname{}, but we found
that the quality of the code was unacceptably low for the standards of
\sysname{}.  As a result Tarn W. Burton created a completely new
library for pretty printing, called \inravina{}%
\footnote{https://github.com/s-expressionists/Inravina}.  \inravina{}
is perfectly integrated with \incless{}, described in
\refSec{sec-printer}.

\section{Format}
\label{sec-format}

\footnote{https://github.com/s-expressionists/Invistra}
